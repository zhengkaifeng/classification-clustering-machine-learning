\documentclass[letterpaper,hidelinks]{article}
\usepackage[left=1in,right=1in,top=1in,bottom=1in]{geometry}
\usepackage{amsmath}
\usepackage{amssymb}
\usepackage{graphicx}
\usepackage{listings}
\usepackage{enumitem}
\usepackage{listings}
\usepackage{algorithm}
\usepackage{algorithmic}
\usepackage{hyperref}
\usepackage{subfigure}
\lstset{language=R,%
	basicstyle=\footnotesize,
	%basicstyle=\color{red},
	breaklines=true,%
	showstringspaces=false,%without this there will be a symbol in the places where there is a space
	numbers=left,%
	numbersep=9pt, % this defines how far the numbers are from the text
	%emph=[2]{word1,word2}, emphstyle=[2]{style},    
}
 
\numberwithin{equation}{section}
\author{KK Feng}
\title{Histogram Clustering with Expectation-Maximization on SAR Images}
\date{}
\begin{document}
\maketitle
\section{Problem}
\subsection{Description}
Image segmentation\footnote{Image segmentation refers to the task of dividing an input image into regions of pixels that belong together.} problem for synthetic aperture radar (SAR) images\footnote{This type of image is well-suited for segmentation by histogram clustering, because the local intensity distributions provide distinctive information about the segments.}.
\subsection{Data}
\begin{itemize}
\item \textbf{histograms.bin}: the histograms extracted from an 800 by 800 grayscale image per following procedure
\begin{enumerate}
\item Select a subset of pixels.
\item Place a rectangle of fixed radius around the site pixel.
\item Select all pixels within the rectangle and sort their intensity values into a histogram.
\end{enumerate}
\end{itemize}
\subsection{Idea}
\begin{itemize}
\item Extract the histograms from the image
\begin{itemize}
\item The histograms were drawn at the nodes of a 4-by-4 pixel grid, therefore there are 200 by 200 = 40000 histograms. 
\item Each histogram was drawn within a rectangle of edge length 11 pixels, so each histogram contains 11 by 11 = 121 values.
\end{itemize}
\item Apply the Expectation-Maximization algorithm and a finite mixture of multinomial distributions.
\end{itemize}

\section{Solution}
\subsection{Plot}
We have an integer $K$ which specifies the number of clusters and a threshold parameter $\tau$ which is used for the termination of the iteration. For $K=3,4,5$, we tested $\tau$ for following values
\begin{align}
1,~0.1,~0.01,~0.001,~0.0001,~0.00001,0.000001,0.0000001
\end{align}
Then we visualize the clustering results, which are the numbers of clusters assigned to the histograms, as an image. Note that when use $image$ function in R, we need to rotate the axes to get the images to the right position. From the figures below, we can see that the larger $K$ is, the smaller $\tau$ the algorithm requires to get a convergent solution.
\begin{figure}
\centering
\subfigure[$\tau$=1]{\includegraphics[width=1.8in]{13}}
\hspace{0.1in}
\subfigure[$\tau$=0.1]{\includegraphics[width=1.8in]{14}}
\\
\subfigure[$\tau$=0.01]{\includegraphics[width=1.8in]{15}}
\hspace{0.1in}
\subfigure[$\tau$=0.001]{\includegraphics[width=1.8in]{16}}
\\
\subfigure[$\tau$=0.0001]{\includegraphics[width=1.8in]{17}}
\hspace{0.1in}
\subfigure[$\tau$=0.00001]{\includegraphics[width=1.8in]{18}}
\\
\subfigure[$\tau$=0.000001]{\includegraphics[width=1.8in]{19}}
\hspace{0.1in}
\subfigure[$\tau$=0.0000001]{\includegraphics[width=1.8in]{110}}
\caption{K=3}
\end{figure}
\begin{figure}
\centering
\subfigure[$\tau$=1]{\includegraphics[width=1.8in]{23}}
\hspace{0.1in}
\subfigure[$\tau$=0.1]{\includegraphics[width=1.8in]{24}}
\\
\subfigure[$\tau$=0.01]{\includegraphics[width=1.8in]{25}}
\hspace{0.1in}
\subfigure[$\tau$=0.001]{\includegraphics[width=1.8in]{26}}
\\
\subfigure[$\tau$=0.0001]{\includegraphics[width=1.8in]{27}}
\hspace{0.1in}
\subfigure[$\tau$=0.00001]{\includegraphics[width=1.8in]{28}}
\\
\subfigure[$\tau$=0.000001]{\includegraphics[width=1.8in]{29}}
\hspace{0.1in}
\subfigure[$\tau$=0.0000001]{\includegraphics[width=1.8in]{210}}
\caption{K=4}
\end{figure}
\begin{figure}
\centering
\subfigure[$\tau$=1]{\includegraphics[width=1.8in]{33}}
\hspace{0.1in}
\subfigure[$\tau$=0.1]{\includegraphics[width=1.8in]{34}}
\\
\subfigure[$\tau$=0.01]{\includegraphics[width=1.8in]{35}}
\hspace{0.1in}
\subfigure[$\tau$=0.001]{\includegraphics[width=1.8in]{36}}
\\
\subfigure[$\tau$=0.0001]{\includegraphics[width=1.8in]{37}}
\hspace{0.1in}
\subfigure[$\tau$=0.00001]{\includegraphics[width=1.8in]{38}}
\\
\subfigure[$\tau$=0.000001]{\includegraphics[width=1.8in]{39}}
\hspace{0.1in}
\subfigure[$\tau$=0.0000001]{\includegraphics[width=1.8in]{310}}
\caption{K=5}
\end{figure}
\subsection{Code}
\lstinputlisting{em.R}
\end{document}